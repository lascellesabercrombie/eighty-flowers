\documentclass[a4paper]{article}
\usepackage[backend=bibtex]{biblatex}
\usepackage[utf8]{inputenc}
\usepackage{soul}
\usepackage{rotating}
\usepackage{ulem}
\usepackage{lineno, blindtext}
\usepackage{etoolbox}
\usepackage{nopageno}
\usepackage{parskip}
\usepackage{xcolor}
\usepackage[paperwidth=6in, paperheight=20in,  top=0.1in, bottom=0.1in, left=0.6in, right=0.4in]{geometry}
\usepackage{tikz}
\usepackage{circledsteps}
\usepackage{stix}
\setlength{\parindent}{0pt}
\preto{\draft}{\resetlinenumber}
\renewcommand{\linenumberfont}{\normalfont\bfseries\small\color{red}}
\setstcolor{red}
\begin{document}
% \colorbox{gray}{
\begin{minipage}{0.1\textwidth}
\Circled{\color{blue}{13}} 
\end{minipage}
\begin{minipage}{0.2\textwidth}
\color{blue}
\null list \#11
b. July 22/75
\end{minipage}
% }
% \colorbox{orange}{
\begin{minipage}{0.7\textwidth}
\begin{flushright}
\color{blue}
\Circled{p.19}\par
Gray
$\rightarrow$
\color{red}\Circled{\color{blue}{Venus's Looking-glass}}\par
\end{flushright}
\end{minipage}
% }
\color{blue}
\vspace{0pt}
% \colorbox{purple}{
\begin{minipage}[t]{0.3\textwidth}
%lh column
Gray: corolla
of perfect flowers
rotate 5-lobed;
fertilized
unopening \&
greatly reduced
flowers in lower
axils; capsule
prismatic, cylindric
\&o awl shaped 3-
locules, opening
by 3 small lateral
valves
above the
middle; axillary
blue \& purplish %"or" in Gray; might be misreading here
flowers earlier
European name
Specul\`aria Speculum- 
Veneris, mirror of V.
was Specularia perfoliata
leaves clasping, about as
broad as long expanded
corollas usually borne
at several upper nodes
capsule pores sub-
median
%LZ takes part on perfoliata from distinction guide rather than longer entry on perfoliata. Use of Gray to do with process of finding out rather than/as much as source of knowledge 

\sout{Taylor}

Bell-looking-glass
(one word
2 hyphens)

only
quotes Century
Onions ???
follow Lindley
he would!
! LZ
\end{minipage}
% }
\hfill
% \colorbox{pink}{
\begin{minipage}[t]{0.6\textwidth}
% \colorbox{yellow}{
\begin{minipage}[t]{0.3\textwidth}
%to the left of this column
\begin{flushright}
    Taylor:\par
\end{flushright}
\vspace{5pt}
\small
Corolla to\par
3/4 across\par
15" H\par
\normalfont
\end{minipage}
% }
% \colorbox{green}{
\begin{minipage}[t]{0.6\textwidth}
Venus's-looking-glass\par
also known as Campanula\par
speculum; leaves alternate\par
margins sometimes toothed to 1 1/2" ?\par
1-3 flower clusters \sout{flwr} deep blue\par
Sometimes white, erect habit\par
{[i.e. "bell" upright]}\par
\end{minipage}
% }
%main rh column
% \colorbox{orange}{
\begin{minipage}[t]{0.9\textwidth}
\ul{Field Flowers} illust.p9 Specularia
perfoliata (Belleflower
f.)
has two kinds of flowers .. earliest
borne on lower part of stems .. insignificant
in appearance .. set seeds without recvg
pollen from other flowers .. the later ones
higher up .. showy .. visited by insects
which carry pollen from one to the other \& bring
about cross-fertilization. Season:
May-Sept. Flowers: solitary or 2 or 3
together .. 5 (rarely 4) reddish-violet petals
annual 6 in to 2ft tall stems rather,
weak Leaves shell-shaped stem-clasping
1" or less D.\par 
\color{red}
? cf. campanula rotundifolia "bluebell-of-
Scotland or bluebell or harebell,
[bell = looking-glass L. Z.] Century:
"many plants take their names from
familiar animal %sic 
without obvious reason"
low herb with delicate drooping blue, bell-
shaped flowers, linear-lanceolate stem-
leaves, those near the root round-heart
shaped or ovate, but early disappearing
.. rarely seen with the flowers.
W.S. \sout{Shak.} Cym.IV 2:[222]: The azur'd hare-bell
like thy veins [] John Lindley 1799[-1865]
(restricted) the spelling harebell
(to) \ul{Scilla nutans} (i.e. the wild
hyacinth or Hyacinthus non-scriptus
["Scotch" 
\color{blue}
$\stackrel{\hbox{(Century Dict)}}{\hbox{$\caretinsert$}}$
\color{red}
"rarely so used in English
works" Cym's British, tho Lindley 
was also English, botanist \& horticulturist
taught at U. of London. Prob. (See
Gray re - Wild Hyacinth affected
by the old Latin name Scilla = squill
\sout{of} lily family for the wild Hyacinth
Scilla Nonscripta = without writing
because the petals are \ul{not} veined like
writing as in the Hyacinthus-Apollo story]
\color{blue}
??? last "comment" mine
- has to be - LZ
\end{minipage}
% }
\end{minipage}
% }
%to the left of rhcolumn
\end{document}