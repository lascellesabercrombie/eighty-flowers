\documentclass[a4paper]{article}
\usepackage[backend=bibtex]{biblatex}
\usepackage[utf8]{inputenc}
\usepackage{soul}
\usepackage{rotating}
\usepackage{ulem}
\usepackage{lineno, blindtext}
\usepackage{etoolbox}
\usepackage{nopageno}
\usepackage{parskip}
\usepackage{xcolor}
\usepackage[paperwidth=6in, paperheight=20in,  top=0.1in, bottom=0.1in, left=0.6in, right=0.4in]{geometry}
\usepackage{tikz}
\usepackage{circledsteps}
\usepackage{stix}
\setlength{\parindent}{0pt}
\preto{\draft}{\resetlinenumber}
\renewcommand{\linenumberfont}{\normalfont\bfseries\small\color{red}}
\setstcolor{red}
\begin{document}
% \colorbox{gray}{
\begin{minipage}{0.1\textwidth}
\Circled{\color{blue}{13}} 
\end{minipage}
\begin{minipage}{0.2\textwidth}
\color{blue}
\null list \#11
b. July 22/75
\end{minipage}
% }
% \colorbox{orange}{
\begin{minipage}{0.7\textwidth}
\begin{flushright}
\color{blue}
\Circled{p.19}\par
Gray
$\rightarrow$
\color{red}\Circled{\color{blue}{Venus's Looking-glass}}\par
\end{flushright}
\end{minipage}
% }
\color{blue}
\vspace{0pt}
% \colorbox{purple}{
\begin{minipage}[t]{0.3\textwidth}
%lh column
\setulcolor{red}
\setul{}{2pt}
Gray: \ul{corolla}
of \ul{perfect} \ul{flowers}
\ul{rotate} \ul{5-lobed};
fertilized
\ul{unopening} \&
greatly reduced
flowers \ul{in} \ul{lower}
\ul{axils}; capsule
prismatic, cylindric
\&o awl shaped 3-
locules, opening
by 3 small lateral
valves
above the
middle; \ul{axillary}
\ul{blue} \& \ul{purplish} %"or" in Gray; might be misreading here
flowers earlier
European name
\setul{}{1pt}
\ul{Specul\`aria Speculum-} 
\ul{Veneris}, mirror of V.
\setul{}{2pt}
was \ul{Spec}
\setul{}{1pt}
\ul{ularia perfoliata}
\setul{}{2pt}
\ul{leaves clasping}, about as
broad as long expanded
corollas usually borne
at several upper nodes
capsule pores sub-
median
%LZ takes part on perfoliata from distinction guide rather than longer entry on perfoliata. Use of Gray to do with process of finding out rather than/as much as source of knowledge 

\sout{Taylor}
\color{red}
\textbf{Bell-looking-glass}
(one word
2 hyphens)

only
quotes Century
Onions ???
\color{blue}
follow Lindley
he would!
! LZ


\end{minipage}
% }
\hfill
% \colorbox{pink}{
\begin{minipage}[t]{0.6\textwidth}
% \colorbox{yellow}{
\begin{minipage}[t]{0.3\textwidth}
%to the left of this column
\begin{flushright}
    Taylor:\par
\end{flushright}
\vspace{5pt}
\small
Corolla to\par
3/4 across\par
15" H\par
\normalfont
\end{minipage}
% }
% \colorbox{green}{
\begin{minipage}[t]{0.6\textwidth}
Venus's-looking-glass\par
\setulcolor{red}
also known as \ul{Campanula}\par
\ul{speculum}; 
\setul{}{2pt}
\ul{leaves} \ul{alternate}\par
margins sometimes toothed to 1 1/2" ?\par
1-3 \ul{flower} \ul{clusters} \sout{flwr} deep blue\par
Sometimes white, \ul{erect} \ul{habit}\par
{[i.e. \ul{"bell"} \ul{upright}]}\par
\end{minipage}
% }
%main rh column
% \colorbox{orange}{
\begin{minipage}[t]{0.9\textwidth}
\ul{Field Flowers}
\setulcolor{red}
illust.p9 \ul{Specularia}
\ul{perfoliata} (Belleflower
f.)
\setul{}{2pt}
has \ul{two} \ul{kinds} \ul{of flowers} .. \ul{earliest}
borne \ul{on} \ul{lower part} \ul{of stems} .. insignificant
in appearance .. set \ul{seeds} \ul{without} recvg
\ul{pollen from} \ul{other} \ul{flowers} .. the \ul{later ones}
\ul{higher up} .. showy .. \ul{visited} by \ul{insects}
\ul{which} carry \ul{pollen} \ul{from} \ul{one} to \ul{the other} \& \ul{bring}
\ul{about} \ul{cross}-\ul{fertilization}. Season:
\ul{May}-\ul{Sept}. Flowers: \ul{solitary} or \ul{2 or} 3
\ul{together} .. 5 (\ul{rarely} 4) \ul{reddish-violet} \ul{petals}
annual 6 in to 2ft \ul{tall stems} \ul{rather}
\ul{weak} \ul{Leaves shell-shaped} \ul{stem}\ul{-clasping}
1" or less D.\par 
\color{red}
? cf. campanula rotundifolia "bluebell-of-
Scotland or bluebell or harebell,
\ul{[bell = looking-glass L. Z.]} 
\setul{}{1pt}
\ul{Century}:
"many plants take their names from
familiar animal %sic 
without obvious reason"
low herb with delicate drooping blue, bell-
shaped flowers, linear-lanceolate stem-
leaves, those near the root round-heart
shaped or ovate, but early disappearing
.. rarely seen with the flowers.
\setulcolor{blue}
W.S. \sout{Shak.} Cym.IV 2:[222]: ul{The azur'd hare-bell}
\ul{like thy veins} [] John Lindley 1799[-1865]
(restricted) the spelling harebell
\setulcolor{red}
(to) \ul{Scilla} \ul{nutans} (i.e. the wild
hyacinth or Hyacinthus non-scriptus
\setulcolor{blue}
\ul{
    ["Scotch"} 
\color{blue}
$\stackrel{\hbox{(Century Dict)}}{\hbox{$\caretinsert$}}$
\color{red}
\ul{"rarely so used in English}
works" \ul{Cym's British}, tho Lindley 
was also English, botanist \& horticulturist
taught at U. of London. Prob. (See
Gray re - Wild Hyacinth affected
by the old Latin name Scilla = squill
\sout{of} lily family for the wild Hyacinth
Scilla Nonscripta = without writing
because the petals are \ul{not} veined like
writing as in the Hyacinthus-Apollo story]
\color{blue}
??? last "comment" mine
- has to be - LZ
\end{minipage}
% }
\end{minipage}
% }
%to the left of rhcolumn
\end{document}