\documentclass[a4paper,9pt]{article}

\usepackage[utf8]{inputenc}
\usepackage{color,soul}
\usepackage{ulem}
\usepackage{blindtext}
\usepackage{rotating}
\usepackage{nopageno}
\usepackage{circledsteps}
\usepackage{stackengine}
\usepackage[paperwidth=5.5in, paperheight=15in, margin=0.25in]{geometry}
%dimensions 5in by 7.75; added .5 and to allow .25 margin
\renewcommand{\labelitemi}{$\textendash$}
\setulcolor{blue}

\begin{document}
\begin{minipage}{0.2\textwidth}
\color{red}
*See p. I A-B
for possible
substitutes
%something has been written over along with ticks - fibit?
\end{minipage}
\begin{minipage}{0.3\textwidth}
\color{black}
\center
80 Flowers
\end{minipage}
\begin{minipage}{0.2\textwidth}
\color{red}
\newsavebox{\abc}
\savebox{\abc}{\parbox{1in}{\ul{Replaced\\
by list\\
Aug 18/75}}}%
\begin{turn}{25}\usebox{\abc}\end{turn}
\end{minipage}
\begin{minipage}{0.1\textwidth}
\flushright
\color{red}
I
\end{minipage}
\begin{enumerate}
\color{black}
\item Aloe\\
(Paget) saw-tooth leaf resembles cactus,
orange flower, closed trumpet\\
\color{red}
\setulcolor{red}
work with [arrow]
\#5 
\ul{AZALEA}
\color{black}
\item \color{red} \Circled[inner ysep=1cm]{\Longstack{{heath f.} $\downarrow$ \color{black}{Andromeda}}}
\color{red}cf or subs. African Lily = Agapanthus
Africanus (Lily-of-the-Nile) or
Agapanthus mooreanus. \sout{Identified Pt J. June 14/75}
See \ul{Taylor}
see also Wood Anemone in 
\ul{Woodland Flowers} illust. color p 56
anemone quinquefolia
\color{black}
\item Anemone\\
\color{red}(
\color{black}
includes pasque-flower
\color{red})\\
\color{black}
\item Arbutus
\color{blue}
(evergreen tree // Ericaceae heath family)\\
%big brackets
\color{black} 
(trailing)
\color{blue}
=herbs
or shrubs ("strictly") (genus \ul{Epigaea}
\ul{repens}
most fragrant wild flower 
also called ground
laurel or winter pink -> or the Mayflower

%Taylor, inc "strictly" shrub
\color{black}
\item Azalea
\color{red}
red \ul{via Jap} \color{blue} \& Korea 
\color{red}= obtusa,
blooms Apr.May
flame A = Calendulacea flowers 5-7 together yellow, orange
or scarlet (leaves broadly elliptic 2-3 " l.) (not fragrant, stamens
sticky-hairy much protruding
%Taylor
(Heath family - inc. arbutus \& 
rhodoendron \ul{see Taylor}
\color{black}
\item baby's breath 
\color{red}
(\sout{See aloe}
or \#61
%this effectively crossed out by arrow coming from #35
subs. \ul{begonia}? or combine? or use orig \# 6 with \#35
\color{black}
?work with buttercup family 
(incl. columbine 
sitfast etc. 
\color{red}\#14
\color{black}
see \ul{cockle} v.2.)
\setulcolor{black}
\item birdseed (see \ul{Gray})


\item bougainvillaea (Paget: trumpet 3-joined
veined leaf-like red petals) {$\stackrel{\hbox{\color{red}{cf\#26}}}{\hbox{\color{blue}{(4 o'clock family)}}}$}
\color{red}
(Rose family) (cinquefoil see note \ul{Weeds} VIII 6/4/75 work with 
pasture rose
\color{black}
\item cactus ("rose-mauve fluted" \& P's)
\color{blue}
finished 
No. 12
\ul{P's} \ul{"red} \ul{top,"} \ul{cardinal} \ul{hat}
\color{black}
\item camellia (soft red berry, looks hard : Bellagio)
%arrow
(See notes \ul{Weeds} 
\setulcolor{blue}
pVIII \& \ul{work} \ul{cinquefoil} with %arrow
 \Circled{Venus's Looking Glass}
subst. cinquefoil (Rose family) Planted Pt. J. 6/4/75

\color{blue}11
%this 11 has lines connecting to Venus's looking glass on different parts of page
\color{black}
\item Campanula (rotundifolia, Scot \ul{bluebell}
 or harebell; cf C - rapunculoides)
\color{red}
orig 
Bellagio
work only if found
\color{blue}
Taylor
184, 1135 

Venus's-looking-glass= Campanula \ul{Speculum} (bellflower family)
\color{red}
Field Flowers p9 illust " belle looking glass
\color{black}
\item clematis (\ul{Crispa} \color{blue}= \color{black} blue jasmine \color{blue}= \color{black} "bluebell"
%or are the quote marks blue too?
or curly : shrubby vine)
\color{blue}
buttercup 
family \color{red}= ranunculus or crowfoot family

[purslane]
%up the side, blue overwritten on red
\color{red}
called Spring 
beauty etc
work with Claytonia Carolinia <-(Pt J 6/24/75)
\tiny
\color{blue}purslane family
\normalsize
\color{red}
was Claytonia virginica using \ul{jackmani}
illust p45 - Woodland flwrs

\ul{clematis jackmani}
hybrid flowers
4-6 inches wide
for contrast only
%bottom right
\end{enumerate}
\end{document}