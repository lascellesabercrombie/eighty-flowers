\documentclass[a4paper]{article}
\usepackage[backend=bibtex]{biblatex}
\usepackage[utf8]{inputenc}
\usepackage{soul}
\usepackage{rotating}
\usepackage{ulem}
\usepackage{lineno, blindtext}
\usepackage{etoolbox}
\usepackage{nopageno}
\usepackage{parskip}
\usepackage{xcolor}
\usepackage[paperwidth=15cm, paperheight=20in,  top=0.1in, bottom=0.1in, left=0.6in, right=0.4in]{geometry}
\usepackage[absolute]{textpos}
\usepackage{tikz}
\usepackage{circledsteps}
\usepackage{stackengine}
\usepackage{stix}
\setlength{\parindent}{0pt}
\preto{\draft}{\resetlinenumber}
\renewcommand{\linenumberfont}{\normalfont\bfseries\small\color{red}}
\setstcolor{red}
\begin{document}
% \colorbox{gray}{
\begin{minipage}{0.1\textwidth}
\Circled{\color{blue}{13}} 
\end{minipage}
\begin{minipage}{0.2\textwidth}
\color{blue}
\small
\null list \#11\par
b. July 22/75
\end{minipage}
% }
% \colorbox{orange}{
\begin{minipage}{0.7\textwidth}
\color{blue}
\begin{flushright}
\Circled{p.19}\par
\end{flushright}
\hspace{1cm}
Gray
$\rightarrow$
\color{red}\Circled{\color{blue}{Venus's Looking-glass}}\par
\end{minipage}

\color{blue}
\begin{minipage}[t]{0.3\textwidth}
%lh column
\setulcolor{red}
\setul{}{2pt}
Gray: \ul{corolla}\par
of \ul{perfect} \ul{flowers}\par
\ul{rotate} \ul{5-lobed};\par
fertilized\par
\ul{unopening} \&\par
greatly reduced\par
flowers \ul{in} \ul{lower}\par
\ul{axils}; capsule\par
prismatic, cylindric\par
+ awl shaped 3-\par
locules, opening\par
by 3 small lateral\par
valves above the\par
middle; \ul{axillary}\par
\ul{blue} + \ul{purplish}\par %"or" in Gray; might be misreading here
flowers earlier\par
European name\par
\setul{}{1pt}
\ul{Specul\`aria Speculum-}\par
\ul{Veneris}, mirror of V.\par
\setul{}{2pt}
was \ul{Spec}
\setul{}{1pt}
\ul{ularia perfoliata}\par
\setul{}{2pt}
\ul{leaves clasping}, about as\par
broad as long expanded\par
corollas usually borne\par
at several upper nodes\par
capsule pores sub-\par
median\par
%LZ takes part on perfoliata from distinction guide rather than longer entry on perfoliata. Use of Gray to do with process of finding out rather than/as much as source of knowledge 

\sout{Taylor}
\color{red}
\textbf{Bell-looking-glass}
(one word
2 hyphens)

only
quotes Century
Onions ???
\color{blue}
follow Lindley
he would!
! LZ
\sout{\& V}
Venus \& Adonis
away .. thru..
empty skies .. queen
and not be seen
(final stanza)
\end{minipage}
\hspace{1cm}
%main rh column
\begin{minipage}[t]{10cm}
\vspace{3cm}
\ul{Field Flowers}
\setulcolor{red}
illust.p9 \ul{Specularia}\par
\savestack{\specularianote}{\Longstack[r]{ {\ul{perfoliata} (Belleflower}}}%
\hspace{3cm}\Longunderstack[r]{{\specularianote}  {f.)} }\par 
\setul{}{2pt}
has \ul{two} \ul{kinds} \ul{of flowers} .. \ul{earliest}\par
borne \ul{on} \ul{lower part} \ul{of stems} .. insignificant\par
in appearance .. set \ul{seeds} \ul{without} recvg\par
\ul{pollen from} \ul{other} \ul{flowers} .. the \ul{later ones}\par
\ul{higher up} .. showy .. \ul{visited} by \ul{insects}\par
\ul{which} carry \ul{pollen} \ul{from} \ul{one} to \ul{the other} \& \ul{bring}\par
\ul{about} \ul{cross}-\ul{fertilization}. Season:\par
\ul{May}-\ul{Sept}. Flowers: \ul{solitary} or \ul{2 or} 3\par
\ul{together} .. 5 (\ul{rarely} 4) \ul{reddish-violet} \ul{petals}\par
annual 6 in to 2ft \ul{tall stems} \ul{rather}\par
\ul{weak} \ul{Leaves shell-shaped} \ul{stem}\ul{-clasping}\par
1" or less D.\par 
\end{minipage}

% absolute positioned boxes

\begin{textblock*}{2cm}(6cm,1.4cm)%
	\begin{minipage}{2cm} 
        \color{blue}
        Taylor:\par
        \vspace{5pt}
        \small
            Corolla to\par
            3/4 across\par
            15" H\par
        \normalfont			
	\end{minipage}%
\end{textblock*}%


\begin{textblock*}{6cm}(8cm,1.4cm)%
	\begin{minipage}{6cm} 
        \color{blue}
		Venus's-looking-glass\par
        \setulcolor{red}
        also known as \ul{Campanula}\par
        \ul{speculum}; 
        \setul{}{2pt}
        \ul{leaves} \ul{alternate}\par
        margins sometimes toothed to 1 1/2" L \par
        1-3 \ul{flower} \ul{clusters} \sout{flwr} deep blue\par
        Sometimes white, \ul{erect} \ul{habit}\par
        {[i.e. \ul{"bell"} \ul{upright}]}\par		
	\end{minipage}%
\end{textblock*}%

\begin{textblock*}{8cm}(6cm,11.5cm)%
	\begin{minipage}{8cm} 
        \color{red}
        ? cf. campanula rotundifolia "bluebell-of-\par
        Scotland or bluebell or harebell,\par
        
        \ul{[bell = looking-glass L. Z.]}\par
        \setul{}{0.5pt}
        \ul{Century}:
        "many plants take their names from\par
        familiar animal %sic 
        without obvious reason"\par
        low herb with delicate drooping blue, bell-\par
        shaped flowers, linear-lanceolate stem-\par
        leaves, those near the root round-heart\par
        shaped or ovate, but early disappearing\par
        .. rarely seen with the flowers.\par
        \setulcolor{blue}
        W.S. \sout{Shak.} Cym.IV 2:[222]: ul{The azur'd hare-bell}\par
        \ul{like thy veins} [] John Lindley 1799[-1865]\par
        (restricted) the spelling harebell\par
        \setulcolor{red}
        (to) \ul{Scilla} \ul{nutans} (i.e. the wild\par
        hyacinth or Hyacinthus non-scriptus\par
        \setulcolor{blue}
        \ul{
            ["Scotch"} 
        \color{blue}
        $\stackrel{\hbox{(Century Dict)}}{\hbox{$\caretinsert$}}$
        \color{red}
        \ul{"rarely so used in English}\par
        works" \ul{Cym's British}, tho Lindley\par
        was also English, botanist \& horticulturist\par
        taught at U. of London. Prob. (See\par
        Gray re - Wild Hyacinth affected\par
        by the old Latin name Scilla = squill\par
        \sout{of} lily family for the wild Hyacinth\par
        Scilla Nonscripta = without writing\par
        because the petals are \ul{not} veined like\par
        writing as in the Hyacinthus-Apollo story]\par
        \color{blue}
        ??? last "comment" mine
        - has to be - LZ	
	\end{minipage}%
\end{textblock*}%
\end{document}