\documentclass[a4paper,10pt]{article}

\usepackage[utf8]{inputenc}
\usepackage{color,soul}
\usepackage{ulem}
\usepackage{blindtext}
\usepackage{rotating}
\usepackage{nopageno}
\usepackage{circledsteps}
\usepackage{latexsym}
\usepackage{parskip}
\usepackage{xcolor}
\usepackage[paperwidth=5.5in, paperheight=10.25in, margin=0.7in]{geometry}
%dimensions 5in by 7.75; I have altered to bring in line with p.20 layout]
\renewcommand{\labelitemi}{$\textendash$}
\setulcolor{red}

\begin{document}
\color{blue}
\begin{flushright}
\Circled{Notes \& orig. drafts (?) page 1}\par
\end{flushright}
\begin{flushleft}
\color{red} 
\begin{turn}{25}%
* See \ul{Taylor p215} 
\end{turn}
\ul{80 Flowers} (possible substitutes\par
\color{red}for the 80 listed on ppI-VI) \hfill \color{blue} \Circled{{$\stackrel{\hbox{orig p \#}}{\hbox{IA-B}}$}}\par

\color{red}
\small
\Circled{3} perennial .. weedy .. Eurasian .. poppy family .. {$\stackrel{\hbox{called also}}{\hbox{killwort sightwort}}$}\par
%above lines not great at the moment in making clear where LZ starts new line for space, squeezes in a bit on the right etc
\Circled{2} $\rightarrow$ the lesser celandine or pilewort, still used in home med. for piles\par
``occurs locally in Eastern states.\par
\begin{itemize}
\color{blue}
\setulcolor{blue}
\normalsize
\item Celandine, \ul{Chelidonium majus} (\ul{Kamm}\par
\tiny 
\color{red}
swallowort - Lewis \& Short\par
\color{blue}
\normalsize
\color{red}
\Circled{$\stackrel{\hbox{{*See:}}}{\hbox{\color{blue}{pp.40-42}}}$} \color{blue}Port Jefferson planted Oct 18/74)\par
%pp.40-42 circled in red below *See:
\color{red}
\small
\Circled{1} (Wordsworth's poems refer to \ul{Ranunculus ficaria}, a crowfoot.\par
see \# 65 vi pV 
%entry for 65 is on pV of working notebook, but gives sage
\color{blue}
$\uparrow$ L = little frog (Buttercup \ i.e. meadow habit
%arrow should point to Ranunculus
\rule{10cm}{0.01cm}
\normalsize
\item Lavender cotton, \ul{Santolina chamaecypar-} \par
\ul{-issus} \ul{Kamm} p 206 illust \#25 Pt. J, planted\par 
%looks v like a new entry for -issus but clearly cont from previous line
\color{red}
\tiny
\newsavebox{\abc}
\savebox{\abc}{
    \fcolorbox{blue}{white}{
    \begin{minipage}{0.5in}
        How great fallen are the great L.Z.
    \end{minipage}
    }
    }
\begin{turn}{25}\usebox{\abc}\end{turn}
%a 2 and 3 in there? presumably an LZ quotation
\color{blue}
\normalsize
Oct 18/74
\small
\color{red}
John Parkinson 1629 \ul{Paradisus Terrestis} ``The rarity \& novelty of this herb, being for the most part but in the gardens of great persons, doth cause it to be of great regard"
%LZ tweaking the quotation?
\color{blue}
\normalsize
\item Spearmint (Pt. J. planted in Oct 74)
\item \ul{Spiderplant} (lily family)
\item \ul{thyme (time)} (Pt. J. {$\stackrel{\hbox{Aug-}}{\hbox{Sep}}$}74) (see \#70 pVI v.i)
\item (\ul{mint geranium} or costmary Pt. J Oct 26/74 
Chyrsanthemum Balsonata GardenEncy Taylor)
\rule{10cm}{0.01cm}
%this item crossed out; probably Balsamita, but doesn't quite look like it; line down to
\begin{flushright}
% may want an alternative arrow to make clear connection / complete line to material above    
$\downarrow$(identified as follow at {$\stackrel{\hbox{Suffolk}}{\hbox{Rose Soc.}}$})
\end{flushright}
\item sweet-fern (Comptonia, \ul{Taylor}; resembles spleenwort (a fern) but \ul{sweet-fern} is a shrub of the walnut family - see Gray)
Oct 28/74 Pt..J.\par
\color{red}
\footnotesize
Taylor lists wild geranium under \ul{geraniaceae} -/a separate family 
Gray p817-818 %column
under g\'eum (Rose family)
%Gray has Geum with grave, but LZ's looks acute
\color{blue}
\normalsize
\item wild geranium (\ul{Taylor}, Geranicum maculatum; also called alumsroot \& chocolateflower\par
\color{red}
but (is) not\par
\ul{Heuchera}\par
also called\par
 alumroot\par
(saxifrage\par
family)
\color{blue}
[the relation to those]{$\stackrel{\hbox{identified*}}{\hbox{Oct 30/74}}$}\par
*neighbor called it wild strawberry, but\par 
no berry\par
%should have square brackets from [*neighbor to berry] but was confusing LaTeX
See Gray - The name of a nymph etc orig a water plant 
?Balkelot? - sour gum, pepperidge \color{red} (\ul{Cheyney} What Tree is That? [identified])
\color{blue}Nyssa sylvatica Pt J 306 E Brway Nov 3/74?


\item Catnip - Same as \underline{Glecoma} hederacea (Taylor)\par
\setulcolor{red}
?\#29 p9 III v.i. ``The \ul{catnip and the} \par
\ul{amaranth! - man's} \ul{earthly household} \par
\ul{peace} \& \ul{the ever-encroa}ch\ul{ing appetite} \ul{for God}"\par
\setulcolor{blue}
Melville, \ul{Pierre} Bk xxv IV (p356 \ul{Signet} ed.)
\color{red}
(unwithering
unfading) %with arrow, pointing to amaranth
\color{red}
catnip - N\'epeta %more accents 
Catmint .. per. \& ann. herbs .. \sout{tall}
\sout{to erect or trailing} tall \& erect, or dwarf trailing, generally (over)
\end{itemize}
\end{flushleft}


\end{document}